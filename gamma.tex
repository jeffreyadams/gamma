%edited starting 8/27/17 for re-submission to Duke
\documentclass[10pt,leqno]{article} 
\usepackage{verbatim} 
\usepackage{amssymb}
\usepackage{mathtools}
\usepackage{amsrefs}
\usepackage{rotating}
\usepackage{amsmath}
%\usepackage{showkeys}
\usepackage{tabularx}
\setlength\extrarowheight{4pt}   %spacing in tables
\usepackage{theorem}
\usepackage[matrix,tips,frame,color,line,poly,curve]{xy}
\renewcommand{\labelenumi}{(\arabic{enumi})}
\newcommand\kappaarrow[2]{#1\overset\kappa\rightarrow#2}
\newtheorem{theorem}[equation]{Theorem}
\newtheorem{corollary}[equation]{Corollary}
\newtheorem{definition}[equation]{Definition}
\newtheorem{lemma}[equation]{Lemma}
\newtheorem{desideratum}[equation]{Desideratum}
\newtheorem{conjecture}[equation]{Conjecture}
\newtheorem{proposition}[equation]{Proposition}
\newtheorem{remark}[equation]{Remark}
{\theorembodyfont{\rmfamily}\newtheorem{theoremplain}[equation]{Theorem}
\newtheorem{remarkplain}[equation]{Remark}
\newtheorem{editorialremarkplain}[equation]{Editorial Remark}
\newtheorem{exampleplain}[equation]{Example}
\newtheorem{corollaryplain}[equation]{Corollary}
\newtheorem{mytable}[equation]{Table}
}

\renewcommand{\sec}[1]{\section{#1}
\renewcommand{\theequation}{\thesection.\arabic{equation}}
  \setcounter{equation}{0}}
\newcommand{\subsec}[1]{\subsection{#1}
\renewcommand{\theequation}{\thesubsection.\arabic{equation}}
  \setcounter{equation}{0}}

\newcommand{\subsubsec}[1]{\subsubsection{#1}
\renewcommand{\theequation}{\thesubsection.\arabic{equation}}
  \setcounter{equation}{0}}

% Danger, Will Robinson!
\def\danger{\begin{trivlist}\item[]\noindent%
\begingroup\hangindent=3pc\hangafter=-2%\clubpenalty=10000%
\def\par{\endgraf\endgroup}%
\hbox to0pt{\hskip-\hangindent\dbend\hfill}\ignorespaces}
\def\enddanger{\par\end{trivlist}}

\newcommand{\Gext}{\negthinspace\negthinspace\phantom{a}^\delta G}
\newcommand{\thetaG}{\negthinspace\negthinspace\phantom{a}^\theta
  G(\C)}
\newcommand{\thetaK}{\negthinspace\negthinspace\phantom{a}^\theta K(\C)}
\newcommand{\qed}{\hfill $\square$ \medskip}
\newenvironment{proof}[1][Proof]{\noindent\textbf{#1.} }{\qed}
\newcommand\exact[3]{1\rightarrow #1\rightarrow #2\rightarrow #3\rightarrow1}
\newcommand{\Aut}{\mathrm{Aut}}
\newcommand{\Inv}{\mathrm{Invol}}
\newcommand{\sgn}{\mathrm{sgn}}
\newcommand{\diag}{\mathrm{diag}}
\newcommand{\gr}{\mathrm{gr}}
\newcommand{\Out}{\mathrm{Out}}
\newcommand{\Int}{\mathrm{Int}}
\renewcommand{\int}{\mathrm{int}}
\newcommand{\Hom}{\mathrm{Hom}}
\newcommand{\Ind}{\mathrm{Ind}}
\newcommand{\Stab}{\mathrm{Stab}}
\newcommand{\Ad}{\mathrm{Ad}}
\newcommand{\zinv}{\mathrm{inv}}
\newcommand{\SRF}{\mathrm{SRF}}
\newcommand{\Gad}{G_\mathrm{ad}}
\newcommand{\Gsc}{G_\mathrm{sc}}
\newcommand{\Zsc}{Z_\mathrm{sc}}
\newcommand{\Ztor}{Z_\mathrm{tor}}
\newcommand{\Gbar}{\overline G}
\newcommand{\Kad}{K_\mathrm{ad}}
\newcommand{\Gal}{\mathrm{Gal}}
\newcommand{\Norm}{\mathrm{Norm}}
\newcommand{\Cent}{\mathrm{Cent}}
\newcommand{\I}{\mathcal I}
\renewcommand{\O}{\mathcal O}
\newcommand{\R}{\mathbb R}
\newcommand{\C}{\mathbb C}
\newcommand{\Z}{\mathbb Z}
\newcommand{\W}{\mathbb W}
\newcommand{\Ztwo}{\mathbb Z_2}
\newcommand{\N}{\mathcal N}
\newcommand{\Q}{\mathbb Q}
\newcommand{\E}{\mathbb E}
\newcommand{\G}{G}
\renewcommand{\H}{\mathbb H}
\newcommand{\h}{\mathfrak h}
\renewcommand{\b}{\mathfrak b}
\renewcommand{\sl}{\mathfrak s\mathfrak l}
\renewcommand{\P}{\mathcal P}
\renewcommand{\a}{\mathfrak a}
\newcommand{\zk}{\mathfrak z_\mathfrak k}
\newcommand{\A}{\mathbb A}
\newcommand{\K}{\mathcal K}
\newcommand{\B}{\mathcal B}
\renewcommand{\k}{\mathfrak k}
\newcommand{\p}{\mathfrak p}
\newcommand{\spint}{\widetilde{Spin}}
\newcommand{\ch}[1]{#1^\vee}
\newcommand\sigmaqc{\sigma_{\text{qc}}}
\newcommand\thetaqc{\theta_{\text{qc}}}
\newcommand{\Fgal}{F_{\text{gal}}}
\newcommand{\Falg}{F_{\text{alg}}}
\newcommand{\cl}{\mathit{cl}}
\newcommand{\Lie}{\mathrm{Lie}}
\newcommand{\opp}{\text{-opp}}

\renewcommand{\t}{\mathfrak t}
\newcommand{\su}{\mathfrak s\mathfrak u}
\newcommand{\g}{\mathfrak g}
\newcommand\inv{^{-1}}
\newcommand\wh{\widehat}
\newcommand{\GL}{\text{GL}}
\newcommand{\SL}{\text{SL}}
\newcommand{\SO}{\text{SO}}
\newcommand{\SU}{\text{SU}}
\newcommand{\Spin}{\text{Spin}}
\renewcommand{\a}{\mathrm a}
\renewcommand{\b}{\mathrm b}
\renewcommand{\c}{\mathrm c}
\newcommand{\LG}{\overset{L}{\vphantom{a}}\negthinspace G}
\newcommand{\WR}{W_\R}
\newcommand{\Y}{\widetilde{\mathcal Y}^\vee}
\begin{document}
\title{Gamma factors for real groups}
\author{Jeffrey Adams}
\maketitle

\bibliographystyle{plain}
\bibliography{/home/jda/bibliographies/refs}

\sec{Preliminaries}

Write a one-dimensional representation of $W_\R$ as  $\a(\delta,t)$ where $\delta\in\Z/2\Z$ and $t\in\C$.
$$
\begin{aligned}
\a(\delta,t)(z)&=|z|^t\\
\a(\delta,t)(j)&=(-1)^\delta
\end{aligned}
$$

Write a two-dimensional representation of $W_\R$ as
$\b(n,t)$ with $n\in \Z$ and $t\in\C$:
$$
\begin{aligned}
\b(n,t)&=\Ind_{\C^\times}^{W_\R}(|z|^t(z/\overline z)^{n/2})\\
&=\Ind_{\C^\times}^{W_\R}(r^te^{in\theta})
\end{aligned}
$$
More explicitly
$$
\begin{aligned}
\b(n,t)(z=re^{i\theta})&=\diag(|z|^t(z/\overline z)^{n/2},|z|^{t}(z/\overline z)^{-n/2})\\
&=\diag(r^te^{in\theta},r^te^{-in\theta})\\
\b(n,t)(j)&=\begin{pmatrix}0&1\\(-1)^n&0\end{pmatrix}
\end{aligned}
$$
Then $\b(n,t)$ is irreducible if and only if $n\ne 0$, and $\b(n,t)\simeq\b(-n,t)$, so
we may assume $n\ge 0$.
Note that 
$$
\b(0,t)\simeq \a(0,t) \oplus \a(1,t)
$$
and
$$
\det(\b(n,t))=a(n\pmod 2,2t)
$$


Note that Knapp [] defines $\phi(\pm,t)$ and $\phi(\ell,t)$ with $t\in\C$ and $\ell\in\Z$. 
The dictionary is
$$
\begin{aligned}
\a(\delta,t)&=\phi((-1)^\delta,t)\\
\b(n,t)&=\phi(n,t/2)\\
\end{aligned}
$$
Note the $t/2$ term. In the other direction
$$
\begin{aligned}
\phi(\epsilon,t)&=\a(\frac{\epsilon-1}2,t)\\
\phi(n,t)&=\b(n,2t)
\end{aligned}
$$

Note that
$$
\begin{aligned}
\a(\delta,t)^*&=\a(\delta,-t)\\
\b(n,t)^*&=\b(n,-t).
\end{aligned}
$$
These follow from the fact that $\theta_{\pi^*}(g)=\theta_{\pi}(g\inv)$. (Note: this is not equal to $\overline{\theta_\pi(g)}$ since 
$W_\R$ is not finite). 

The root numbers $\epsilon(s,\rho,\Psi)$ are defined for $s\in\C$, $\rho$ a representation of $W_\R$, and $\Psi$ a non-trivial additive character of $\R$.
In the case of $\R$ this is independent of $s$. Take $\Psi_0(x)=e^{2\pi ix}$,
and set
$$
\epsilon(\rho)=\epsilon(s,\rho,\Psi_0).
$$
From Knapp [] we have the following formula for $\epsilon$.

$$
\begin{aligned}
\epsilon(\a(\delta,t))&=i^\delta\\
\epsilon(\b(n,t))&=i^{|n|+1}
\end{aligned}
$$

Let $\Gamma(z)$ be the standard $\Gamma$-function. Here are some properties we need.
First of all
$$
\Gamma(z+1)=z\Gamma(z)\quad \Gamma(z)=\frac{\Gamma(z+1)}z
$$
and
$$
\Gamma(z+n)=z^{\overline n}\Gamma(z)\quad (n\in \Z_{\ge 0}).
$$
Furthermore
\begin{equation}
\label{e:ref}
z+w=1\Rightarrow \Gamma(z)\Gamma(w)=\frac{\pi}{\sin(\pi z)}=\frac{\pi}{\sin(\pi w)}
\end{equation}
and
\begin{equation}
\label{e:dup}
w-z=\frac12\Rightarrow \Gamma(z)\Gamma(w)=2^{1-2z}\sqrt\pi \Gamma(2z)
\end{equation}


\newpage
\begin{lemma}
\label{l:Gammas}
If $z+w=\frac12$ then
$$
\begin{aligned}
  \frac{\Gamma(z)}{\Gamma(w)}&=2^{1-2z}\pi^{-\frac12}\Gamma(2z)\cos(\pi z)\\
  &=\frac1{2^{1-2w}\pi^{-\frac12}\Gamma(2w)\cos(\pi w)}
\end{aligned}  
$$


If $z+w=\frac32$ then
$$
\begin{aligned}
\frac{\Gamma(z)}{\Gamma(w)}&=-2^{2-2z}\pi^{-\frac12}\Gamma(2z-1)\cos(\pi z)\\
&=\frac1{-2^{2-2w}\pi^{-\frac12}\Gamma(2w-1)\cos(\pi w)}
\end{aligned}
$$

Suppose $z+w=\frac 12+\delta$ with $\delta=0,1$. Then
$$
\begin{aligned}
\frac{\Gamma(z)}{\Gamma(w)}&=(-2)^\delta2^{1-2z}\pi^{-\frac12}\Gamma(2z-\delta)\cos(\pi z)\\
&=\frac1{(-2)^\delta2^{1-2w}\pi^{-\frac12}\Gamma(2w-\delta)\cos(\pi w)}
\end{aligned}
$$

\end{lemma}

\begin{proof}
$$
\begin{aligned}
\frac{\Gamma(z)}{\Gamma(w)}&=
\frac{\Gamma(z)\Gamma(z+\frac12)}{\Gamma(w)\Gamma(z+\frac12)}\\
&=
\frac{2^{1-2z}\sqrt\pi\Gamma(2z)}
{\frac\pi{\sin(\pi(z+\frac12))}}
\end{aligned}
$$
where we've applied  \eqref{e:dup} to the numerator, and \eqref{e:ref} to the denominator, using $w+(z+\frac12)=(w+z)+\frac12=\frac12+\frac12=1$.

Or course the second assertion follows by symmetry. Let's check the equality directly:
$$
\begin{aligned}
2^{1-2z}\pi^{-\frac12}\Gamma(2z)\cos(\pi z)&*2^{1-2w}\pi^{-\frac12}\Gamma(2w)\cos(\pi w)
=2^{2-2(z+w)}\pi\inv \Gamma(2z)\Gamma(2w)\cos(\pi z)\cos(\pi w)\\
&=
2^{2-2(\frac12)}\pi\inv \Gamma(2z)\Gamma(1-2z)\cos(\pi z)\cos(\pi(\frac12-z)\\
&=2\pi\inv \frac\pi{\sin(2\pi z)}\cos(\pi z)\sin(\pi z)=1
\end{aligned}
$$

For the next assertion, with $z+w=\frac 32$ compute:
$$
\begin{aligned}
  \frac{\Gamma(z)}{\Gamma(w)}&=
  \frac{\Gamma(z)\Gamma(z-\frac12)}{\Gamma(w\Gamma(z-\frac12))}\\
  &=
  \frac{2^{1-2(z-\frac12)}\sqrt\pi \Gamma(2z-1)}{\frac\pi{\sin(\pi(z-\frac12))}}
  \\
  &=2^{2-2z}\pi^{-\frac12}\Gamma(2z-1)\sin(\pi(z-\frac12)\\
    &=-2^{2-2z}\pi^{-\frac12}\Gamma(2z-1)\cos(\pi z))\\
\end{aligned}
$$

  \end{proof}







Here are the formulas for the normalized $L$-functions.
$$
\begin{aligned}
L(s,\a(\delta,t))&=\pi^{-\frac{s+t+\delta}2}\Gamma(\frac{s+t+\delta}2)\quad(\delta=0,1)\\
L(s,\b(n,t)&=2(2\pi)^{-(s+\frac{t+ |n|}2)}\Gamma(s+\frac{t+|n|}2)
\end{aligned}
$$
The $\gamma$ factor satisfies:
$$
\gamma(s,\rho,\Psi)=\epsilon(s,\rho,\Psi)\frac{L(1-s,\rho^*)}{L(s,\rho)}
$$

Let's compute $\gamma(\a(\delta,t))$.
\begin{equation}
\begin{aligned}
\gamma(s,\a(\delta,t))&=i^\delta\frac{L(1-s,\a(\delta,-t))}{L(s,\a(\delta,t))}\\
&=
i^\delta
\frac{\pi^{-\frac{(1-s)-t+\delta}2}\Gamma(\frac{1-s-t+\delta}2)}
{\pi^{-\frac{s+t+\delta}2}\Gamma(\frac{s+t+\delta}2)}\\
&=i^\delta
\pi^{-\frac 12+s+t}\frac{\Gamma(\frac{1-s-t+\delta}2)}{\Gamma(\frac{s+t+\delta}2)}
\end{aligned}
\end{equation}

Now consider  $\b(n,t)$:
$$
\begin{aligned}
\gamma(s,\b(n,t))&=i^{|n|+1}\frac{L(1-s,\b(n,-t))}{L(s,\b(n,t))}\\
&=i^{|n|+1}
\frac{2(2\pi)^{-(1-s+\frac{-t+|n|}2)}\Gamma((1-s)+\frac{-t+|n|}2)}{
2(2\pi)^{-(s+\frac{t+|n|}2)}\Gamma(s+\frac{t+|n|}2)}\\
&=i^{|n|+1}(2\pi)^{-1+2s+t}
\frac{\Gamma(1-s+\frac{-t+|n|}2)}{\Gamma(s+\frac{t+|n|}2)}
\end{aligned}
$$

Here is a summary:
\begin{lemma}
  \label{l:gammas}
$$
  \gamma(s,\a(\delta,t))=\pi^{-\frac 12+s+t}\frac{\Gamma(\frac{1-s-t+\delta}2)}{\Gamma(\frac{s+t+\delta}2)}
$$
  and
  $$
  \gamma(s,\b(n,t))=i^{|n|+1}(2\pi)^{-1+2s+t}\frac{\Gamma(1-s+\frac{-t+|n|}2)}{\Gamma(s+\frac{t+|n|}2)}
  $$
\end{lemma}


Note that in the formula for $\gamma(s,\a(\delta,t))$,
$\frac{1-s-t+\delta}2+\frac{1+s+t+\delta}2=\frac12+\delta\in\frac12,\frac32$, so we can apply Lemma \ref{l:Gammas}.
This gives a formula of a different form:

\begin{lemma}
$$
\begin{aligned}
\gamma(s,\a(\delta,t))&=(-i)^\delta\pi\inv (2\pi)^{s+t}\Gamma(1-s-t)\cos(\pi(\frac{1-s-t}2+\frac\delta 2))\\
&=\frac1{i^\delta\pi\inv(2\pi)^{1-s-t}\Gamma(s+t)\cos(\pi(\frac{s+t}2+\frac\delta 2))}
\end{aligned}
$$
\end{lemma}


\begin{proof}
  $$
  \begin{aligned}
    \gamma(s,\a(\delta,t))&=i^\delta\pi^{-\frac 12+s+t}\frac{\Gamma(\frac{1-s-t+\delta}2)}{\Gamma(\frac{s+t+\delta}2)}\\
    &= i^\delta\pi^{-\frac 12+s+t}(-2)^\delta 2^{1-2\frac{1-s-t+\delta}2}\pi^{-\frac12}\Gamma(2\frac{1-s-t+\delta}2-\delta)\cos(\pi( \frac{1-s-t+\delta}2))\\
    &=    i^\delta\pi^{-1+s+t}(-2)^\delta2^{s+t-\delta}\Gamma(1-s-t)\cos(\pi( \frac{1-s-t+\delta}2))\\
    &=    (-i)^\delta\pi^{-1+s+t}2^{s+t}\Gamma(1-s-t)\cos(\pi( \frac{1-s-t+\delta}2))\\
        &=    (-i)^\delta\pi\inv(2\pi)^{s+t}\Gamma(1-s-t)\cos(\pi (\frac{1-s-t+\delta}2))
  \end{aligned}
  $$
  The second equality is similar. 
\end{proof}

Sanity check: the equality of the two expressions can be checked directly:
$$
\begin{aligned}
  (-i)^\delta\pi\inv& (2\pi)^{s+t}\Gamma(1-s-t)\cos(\pi(\frac{1-s-t}2+\frac\delta 2))*
  i^\delta\pi\inv(2\pi)^{1-s-t}\Gamma(s+t)\cos(\pi(\frac{s+t}2+\frac\delta 2))\\
  &=\pi^{-2}(2\pi) \Gamma(1-s-t)\Gamma(s+t)
  \cos(\pi(\frac{1-s-t}2+\frac\delta 2))\cos(\pi(\frac{s+t}2+\frac\delta 2))\\
  &=  \frac2\pi\frac{\pi}{\sin(\pi(s+t))}\cos(\pi(\frac{s+t-\delta}2-\frac12)\cos(\pi(\frac{s+t}2+\frac\delta 2))\\
  &=  \frac{2\sin(\pi(\frac{s+t}2-\frac\delta2))\cos(\pi(\frac{s+t}2+\frac\delta2))}{2\cos(\pi(\frac{s+t}2)\sin(\pi(\frac{s+t}2))}\\
  &=\begin{cases}1&\delta=0\\
  \frac{(-1)\cos(\pi(\frac{s+t}2))(-1)\sin(\pi(\frac{s+t}2))}{\cos(\pi(\frac{s+t}2)\sin(\pi(\frac{s+t}2)))}=1&\delta=1
  \end{cases}
\end{aligned}
$$


\begin{lemma}
$$ \gamma(s,\a(\delta,t_1))*\gamma(s,\a(\delta,t_2))=(-1)^\delta(2\pi)^{-1+2s+t_1+t_2}\frac{\Gamma(1-s-t_1)}{\Gamma(s+t_2)}\frac{\cos(\pi(\frac{1-s-t_1}2+\frac\delta2)}
  {\cos(\pi(\frac{s+t_2}2+\frac\delta2)}
  $$
  Suppose
  $$
  t_1-t_2\in 2\Z+1
  $$
  Then
$$ \gamma(s,\a(\delta,t_1))*\gamma(s,\a(\delta,t_2))=(2\pi)^{-1+2s+t_1+t_2}(-1)^{\frac{t_1-t_2-1}2}\frac{\Gamma(1-s-t_1)}{\Gamma(s+t_2)}
$$
Note this is independent of $\delta$.

  In particular suppose
  $$
t\in \Z+\frac12.
  $$
  Then
$$
  \gamma(s,\a(\delta,t))*\gamma(s,\a(\delta,-t))=(2\pi)^{-1+2s}(-1)^{t-\frac12}\frac{\Gamma(1-s-t)}{\Gamma(s-t)}
  $$
\end{lemma}

\begin{proof}
The first statement follows from the previous Lemma, applying the first and second identity once each:
$$
\begin{aligned}
  \gamma(s,\a(\delta,t_1))*\gamma(s,\a(\delta,t_2))&=
\frac{(-i)^\delta\pi\inv (2\pi)^{s+t_1}\Gamma(1-s-t_1)\cos(\pi(\frac{1-s-t_1}2+\frac\delta 2))}
{i^\delta\pi\inv(2\pi)^{1-s-t_2}\Gamma(s+t_2)\cos(\pi(\frac{s+t_2}2+\frac\delta 2))}\\
&=(2\pi)^{-1+2s+t_1+t_2}\frac{\Gamma(1-s-t_1)}{\Gamma(s+t_2)}\frac{\cos(\pi(\frac{1-s-t_1}2+\frac\delta 2))}{\cos(\pi(\frac{s+t_2}2+\frac\delta 2))}
\end{aligned}
$$

For the second use:
$$
\begin{aligned}
\frac{\cos(\pi(\frac{1-s-t_1+\delta}2))}
     {\cos(\pi(\frac{s+t_2+\delta}2))}&=
     \frac{\cos(\pi(\frac{s+t_1-\delta-1}2))}
          {\cos(\pi(\frac{s+t_2+\delta}2))}\\
          &=
    \frac{\cos(\pi(\frac{s+t_2+\delta}2+\frac{t_1-t_2-1}2-\delta))}
       {\cos(\pi(\frac{s+t_2+\delta}2))}\\
\end{aligned}
$$
If $\frac{t_1-t_2-1}2-\delta\in\Z$ this is equal to
$(-1)^{\frac{t_1-t_2-1}2-\delta}$.

For the last identity take $t_1=t,t_2=-t$. The condition is $t_1-t_2\in 2\Z+1$, i.e.
$2t\in 2\Z+1$.
\end{proof}

\sec{Some identities}

It is convenient to change notation and for $x,y\in \C, x-y\in\Z$, define
$$
\c(x,y)=\b(x-y,x+y).
$$
Note that
$$
\c(x,y)=\c(y,x)
$$
and also
$$
\b(x,y)=\c(\frac{x+y}2,\frac{-x+y}2)
$$
Therefore
$$
\epsilon(\c(x,y))=i^{|x-y|+1}
$$
and
$$
\gamma(\c(x,y))=i^{|x-y|+1}(2\pi)^{-1+2s+x+y}\frac{\Gamma(1-s+\frac{-|x+y|+x-y}2)}{\Gamma(s+\frac{|x+y|+x-y}2)}
$$
In particular, we may assume $x\ge y$, in which case
$$
\gamma(\c(x,y))=i^{x-y+1}(2\pi)^{-1+2s+x+y}\frac{\Gamma(1-s-y)}{\Gamma(s+x)}
$$



\begin{lemma}
Suppose $t_1-t_2\in2\Z+1$. Then
$$
\gamma(s,a(0,t_1))\gamma(s,a(0,t_2))=
\gamma(s,a(1,t_1))\gamma(s,a(1,t_2)).
$$
Also for $\delta=0,1$:
$$
\gamma(s,\a(\delta,t_1))\gamma(s,\a(\delta,t_2))=\gamma(s,\c(t_1,t_2))
$$
\end{lemma}


See Proposition \ref{p:basic}.

\begin{lemma}
  Assume $x-y\in2\Z$. Then
$$
  \gamma(s(\a,(\delta,x))\gamma(s,\c(y,z))=
  \gamma(s(\a,(\delta,y))\gamma(s,\c(x,z))
  $$
\end{lemma}

\begin{proof}
  Turn the crank\dots
  We want to know that
$$
  (-i)^\delta\pi\inv(2\pi)^{s+x}\Gamma(1-s-x)\cos(\pi(\frac{1-s-x+\delta}2))
  i^{y-z+1}(2\pi)^{-1+2s+y+z}\frac{\Gamma(1-s-z)}{\Gamma(s+y)}
  $$
  and
$$
(-i)^\delta\pi\inv(2\pi)^{s+y}\Gamma(1-s-y)\cos(\pi(\frac{1-s-y+\delta}2))
i^{x-z+1}(2\pi)^{-1+2s+x+z}\frac{\Gamma(1-s-z)}{\Gamma(s+x)}
$$
After some obvious cancellations we need to show
$$
i^y\frac{\Gamma(1-s-x)}{\Gamma(s+y)}\cos(\pi(\frac{1-s-x+\delta}2))
=
i^x\frac{\Gamma(1-s-y)}{\Gamma(s+x)}\cos(\pi(\frac{1-s-y+\delta}2))
$$
Rearranging terms leads to
$$
i^y\Gamma(1-s-x)\Gamma(s+x)\cos(\pi(\frac{1-s-x+\delta}2))
=
i^x\Gamma(1-s-y)\Gamma(s+y)\cos(\pi(\frac{1-s-y+\delta}2))
$$
Apply \eqref{e:ref} to the left hand side:
$$
\begin{aligned}
  i^y\Gamma(1-s-x)\Gamma(s+x)\cos(\pi(\frac{1-s-x+\delta}2))
&=
i^y\pi\frac{\cos(\pi(\frac{1-s-x+\delta}2)}{\sin(\pi(1-s-x))}\\
&=
i^y\pi\frac{\cos(\pi(\frac{1-s-x+\delta}2)}{2\sin(\pi(\frac{1-s-x}2))\cos(\pi(\frac{1-s-x}2))}
\end{aligned}
$$
If $\delta=0$ this equals
$$
\frac{i^y\pi}{2\sin(\pi(\frac{1-s-x}2))}
$$
If $\delta=1$ the numerator is equal to $-\sin(\pi(\frac{1-s-x}2)$, so this equals
$$
\frac{-i^y\pi}{2\cos(\pi(\frac{1-s-x}2))}
$$
Exactly the same applies to the right hand side. So, if $\delta=0$ we need to prove
$$
\frac{i^y\pi}{2\sin(\pi(\frac{1-s-x}2))}
=
\frac{i^x\pi}{2\sin(\pi(\frac{1-s-y}2))}
$$
i.e.
$$
\frac{\sin(\pi(\frac{1-s-y}2))}
     {\sin(\pi(\frac{1-s-x}2))}=i^{x-y}
     $$
     which is equivalent to
$$
\frac{\sin(\pi(\frac{1-s-x}2+\frac{x-y}2))}
{\sin(\pi(\frac{1-s-x}2))}=i^{x-y}
$$
Since $\frac{x-y}2\in\Z$ the left hand side is $(-1)^{\frac{x-y}2}$, which equals the right hand side.

  When $\delta=1$ we have to show
$$
\frac{-i^y\pi}{2\cos(\pi(\frac{1-s-x}2))}
=
\frac{-i^x\pi}{2\cos(\pi(\frac{1-s-y}2))}
$$
which follows similarly.
\end{proof}

\begin{lemma}

  Suppose $x-y\in\Z, z-w\in\Z$ and $y-z\in\Z$. Then
$$
\gamma(s(\c(x,y))\gamma(s(\c(z,w))=
\gamma(s(\c(x,z))\gamma(s(\c(y,w))
$$
\end{lemma}

Note that the assumptions are what is need to make all terms well defined.

\begin{proof}
  We need to show
$$
  i^{x-y+1}i^{z-w+1}(2\pi)^{-1+2s+x+y}(2\pi)^{-1+2s+z+w}\frac{\Gamma(1-s-y)}{\Gamma(s+x)}\frac{\Gamma(1-s-w)}{\Gamma(s+z)}
  $$
  equals
$$
i^{x-z+1}i^{y-w+1}(2\pi)^{-1+2s+x+z}(2\pi)^{-1+2s+y+w}\frac{\Gamma(1-s-z)}{\Gamma(s+x)}\frac{\Gamma(1-s-w)}{\Gamma(s+y)}
$$
After some obvious cancellations and rearranging terms we have to prove
$$
i^{-y+z}\Gamma(1-s-y)\Gamma(s+y)=
i^{-z+y}\Gamma(1-s-z)\Gamma(s+z)
$$
Applying \eqref{e:ref} we have to show
$$
\frac{\pi i^{-y+z}}{\sin(\pi(s+y))}
=
\frac{\pi i^{-z+y}}{\sin(\pi(s+z))}
$$
or equivalently
$$
\frac{\sin(\pi(s+z))}{\sin(\pi(s+y))}=i^{-z+y}i^{y-z}
$$
This is equivalent to
$$
\frac{\sin(\pi((s+y)+(-z+y)))}{\sin(\pi(s+y))}=(-1)^{-z+y}
$$
which proves the result.
\end{proof}

Here is a summary.

\begin{proposition}
\hfil

\begin{enumerate}
\item $\gamma(s,\a(0,t_1))\gamma(s,\a(0,t_2))=\gamma(s,\a(1,t_1))\gamma(s,\a(1,t_2))$ \quad($t_1-t_2\in2\Z+1$)
\item $\gamma(s,\a(\delta,t_1))\gamma(s,\a(\delta,t_2))=\gamma(s,\c(t_1,t_2))$ \quad($t_1-t_2\in2\Z+1$)
\item   $  \gamma(s,\a(\delta,x))\gamma(s,\c(y,z))=\gamma(s,\a(\delta,y))\gamma(s,\c(x,z))$ \quad$(x-y\in2\Z)$
\item 
$\gamma(s,\c(x,y))\gamma(s,\c(z,w))=\gamma(s,\c(x,z))\gamma(s,\c(y,w))$\quad(
$x-y,z-w,y-z\in\Z$)
\end{enumerate}
\end{proposition}

\sec{The cross action}

Let $\LG=\langle \ch G,\ch\delta\rangle$ be the L-group of the inner class. 
Fix a $\ch\delta$-stable Cartan subgroup $\ch H$ of $\ch G$.
Consider the set
$$
\P=\{(y,\gamma)\}
$$
where
\begin{enumerate}
\item $\gamma$ is a semisimple of $\ch\g$
\item $y\in \LG-\ch G$
\item $y^2=\exp(2\pi i\gamma)$
\item $\exp(\gamma-y\gamma)=1$
\end{enumerate}
We've written $y\gamma$ for $\Ad(y)(\gamma)$. 

$$
\Y =\Norm_{\ch G\ch\delta}(\ch H).
$$
Suppose $y\in \Y,\gamma\in X_*(\ch H)\otimes\C$ satisfy:
$$
\begin{aligned}
\gamma-y\gamma&\in X_*(\ch H)\\
y^2&=\exp(2\pi i\gamma)
\end{aligned}
$$
(where $y\gamma=\Ad(y)\gamma$). 

Define a homomorphism $\phi=\phi(y,\gamma):\WR\rightarrow \LG$ by:
$$
\begin{aligned}
\phi(z)&=z^\gamma\overline z^{y\gamma}\\
\phi(j)&=e^{-\pi i\gamma}y
\end{aligned}
$$
The first formula is shorthand for
$$
\phi(e^z)=e^{z\lambda -\overline zy\lambda}
$$


It is easy to see $\phi$ is a Weil homomorphism: the condition on $\gamma-y\gamma$ insures it is well defined, and
the condition on $y^2$ implies $\phi(j)^2=\phi(-1)$. Furthermore every such homomorphism is $\ch G$-conjugate to one of this form.

The normalizer of $\ch H$ acts on $\Y$ by conjugation, and this factors 


\sec{Representations of the Weil group}

Consider an $n$-dimensional representation $\phi$ of $W_\R$. After conjugating we can write
$\phi$ as a direct sum of terms $\a(\delta,t)$ and $\b(n,\ell)$. 



\enddocument
\end
%%% Local Variables:  
%%% mode: latex
%%% TeY-master: t
%%% End: s_{1}:0 -> -1 6$$
